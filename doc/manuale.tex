\section{Manuale d'uso}
\textbf{Importante:} per la compilazione è richiesto \textbf{Java 11} o superiore.

\subsection{Linux e MacOS}

\paragraph{Compilazione}
Eseguire lo script \texttt{build.sh}, che si occuperà di scaricare la libreria JSON e di compilare il client e il server.

\paragraph{Esecuzione}
Per eseguire il server e il client, eseguire gli script \texttt{run\_server.sh} e \texttt{run\_client.sh}. Se si vuole avviare il server o il client con una configurazione diversa da quella standard, passare il file di configurazione come parametro. \\
\textbf{Esempio}: \texttt{./run\_server.sh config/server.conf}

\subsection{Windows}

\paragraph{Compilazione}
Per compilare il server eseguire il comando dalla cartella \texttt{Turing}:

\begin{verbatim}
javac
  -sourcepath src/
  -classpath lib/<nome della libreria>
  src/turing/server/*.java 
  -d <cartella di output>
\end{verbatim}
Per il client sostituire \texttt{server/*.java} con \texttt{client/*.java}.

\paragraph{Esecuzione}
Sempre dalla cartella \texttt{Turing}, eseguire:

\begin{verbatim}
java
  -classpath <cartella di output>;lib/<nome della libreria>
  -Djava.net.preferIPv4Stack=true
  turing.server.Server <eventuale file di configurazione>
\end{verbatim}
Per il client sostituire \texttt{server.Server} con \texttt{client.Client}.
\medskip \\
\textbf{Importante}: Per separare le cartelle argomento di \texttt{classpath}, su Windows è necessario il punto e virgola, mentre su Linux e su MacOS i due punti.

\subsection{Storia delle versioni}
Codice sorgente sotto licenza MIT. \\
L'intera storia delle versioni è consultabile all'indirizzo: \\
\href{https://github.com/michelezoncheddu/Turing}{\texttt{github.com/michelezoncheddu/Turing}}.
