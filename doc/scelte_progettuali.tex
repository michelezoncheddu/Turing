\section{Scelte progettuali}
Di seguito verranno illustrate le principali scelte progettuali effettuate durante la realizzazione del progetto.

\subsection{Architettura}
\sloppy
L'architettura del server è stata scelta tra due soluzioni, che sono state trattate più in dettaglio durante il Laboratorio di Reti di Calcolatori:

\begin{enumerate}
	\item multithread sincrona con I/O bloccante (\texttt{Sockets} di \texttt{Java IO});
	\item monothread sincrona con I/O non bloccante (\texttt{Selectors} di \texttt{Java NIO}).
\end{enumerate}

Le due soluzioni hanno pregi e difetti complementari, ma il trade-off principale è tra \textbf{velocità} e \textbf{scalabilità}. Ai fini di questo progetto didattico è stata ritenuta più importante la reattività, garantita (sotto carichi non eccessivi) dalla prima soluzione.

\medskip

Per ogni client, il server riserva un thread, che si occuperà di gestire la connessione TCP per l'intero tempo di vita del client.
