\section{Scelte progettuali}
Di seguito verranno illustrate le più importanti scelte progettuali effettuate durante la realizzazione del progetto.

\subsection{Architettura}
\sloppy
Per l'architettura del server erano possibili principalmente due soluzioni, che sono state trattate più in dettaglio durante il Laboratorio di Reti di Calcolatori:
\begin{enumerate}
	\item multithread sincrona con I/O bloccante (\texttt{Sockets} di \texttt{Java IO});
	\item monothread sincrona con I/O non bloccante (\texttt{Selectors} di \texttt{Java NIO}).
\end{enumerate}
Le due soluzioni hanno pregi e difetti complementari, ma il trade-off principale è tra \textbf{velocità} e \textbf{scalabilità}.

Per questo progetto didattico è stata ritenuta più importante la velocità, quindi è stata scelta la prima soluzione.

\subsection{Gestione della concorrenza}
Per ogni client, il server riserva un thread, che si occuperà di gestire l'intera connessione TCP.

I thread sono gestiti da un \textbf{cached thread pool}, scelto per la sua elasticità nella gestione delle risorse; sono inoltre indipendenti ed isolati, e per ogni richiesta esterna devono utilizzare uno dei seguenti manager:
\begin{itemize}
	\item \texttt{userManager}: gestisce concorrentemente gli utenti registrati;
	\item \texttt{documentManager}: gestisce concorrentemente\footnote{In questo caso la corretta gestione della concorrenza non è garantita da una struttura dati concorrente, ma dalle locks delle sezioni.} i documenti salvati;
	\item \texttt{addressManager}: gestisce concorrentemente gli indirizzi IP multicast da assegnare alle chat dei gruppi di lavoro.
\end{itemize}
