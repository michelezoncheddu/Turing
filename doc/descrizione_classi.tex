\section{Descrizione delle classi}
Di seguito una breve descrizione delle classi, suddivise per package. Per approfondire, vedere il Javadoc alla pagina \texttt{index.html} nella cartella \texttt{doc/javadoc/}.

\subsection{Classi condivise}
\begin{itemize}
	\item \texttt{Fields}: per i valori dei campi dei messaggi JSON;
	\item \texttt{UserManagerAPI}: per la registrazione degli utenti;
	\item \texttt{ClientNotificationManagerAPI}, \texttt{ServerNotificationManagerAPI}: per l'invio delle notifiche.
\end{itemize}

\subsection{Classi del server}
Oltre ai tre manager e al confrontatore di indirizzi IP, troviamo le classi:
\begin{itemize}
	\item \texttt{Server}: la classe principale che lo inizializza (caricando ed effettuando il parsing dell'eventuale file di configurazione) e gestisce il dispatch delle connessioni;
	\item \texttt{ClientHandler}: worker che gestisce le connessioni con gli utenti;
	\item \texttt{User}: per memorizzare i dati degli utenti, come password, documenti di proprietà, documenti che è possibile modificare e notifiche pendenti;
	\item \texttt{Document}: per memorizzare tutti i metadati dei documenti, come creatore, utenti ammessi alla modifica e indirizzo della chat di lavoro;
	\item \texttt{Section}: per interagire con la singola sezione del documento, permettendo la modifica da al più un utente alla volta, e leggendo e scrivendo sul disco.
\end{itemize}

\subsubsection{Eccezioni}
\begin{itemize}
	\item \texttt{AlreadyLoggedException}: se l'utente è già loggato;
	\item \texttt{InexistentDocumentException}: se il documento cercato non esiste;
	\item \texttt{InexistentUserException}: se l'utente cercato non esiste;
	\item \texttt{PreExistentDocumentException}: se il documento che si vuole creare esiste già;
	\item \texttt{UserNotAllowedException}: se l'utente non è autorizzato ad interagire col documento.
\end{itemize}

\subsection{Classi del client}
\begin{itemize}
	\item \texttt{ChatListener}: un thread dedicato per la ricezione dei messaggi della chat, creato e distrutto all'occorrenza;
	\item \texttt{Client}: la classe principale: carica ed effettua il parsing dell'eventuale file di configurazione e fa partire il thread per l'interfaccia grafica con il metodo \texttt{invokeLater};
	\item \texttt{ClientGUI}: la classe che si occupa della creazione e della gestione dell'interfaccia grafica;
	\item \texttt{ClientNotificationManager}: per registrarsi al servizio di notifiche;
	\item \texttt{Connection}: per implementare il protocollo request/reply;
	\item \texttt{Document}: rappresenta una versione minimale del documento salvato sul server;
	\item \texttt{Operation}: per implementare la logica del client.
\end{itemize}
